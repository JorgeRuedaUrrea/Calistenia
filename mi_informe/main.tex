\documentclass{article}
\usepackage[utf8]{inputenc}
\usepackage[spanish]{babel}
\usepackage{listings}
\usepackage{graphicx}
\graphicspath{ {images/} }
\usepackage{cite}

\begin{document}

\begin{titlepage}
    \begin{center}
        \vspace*{1cm}
            
        \Huge
        \textbf{CALISTENIA}
            
        \vspace{0.5cm}
        \LARGE
        Marzo 9
            
        \vspace{1.5cm}
            
        \textbf{Jorge Enrique Rueda Urrea}
            
        \vfill
            
        \vspace{0.8cm}
            
        \Large
        Despartamento de Ingeniería Electrónica y Telecomunicaciones\\
        Universidad de Antioquia\\
        Medellín\\
        Marzo de 2021
            
    \end{center}
\end{titlepage}

\tableofcontents
\newpage
\section{Descripcion de la actividad}\label{intro}
Esta actividad consiste en seleccionar un grupo de 3 personas de distintas edades a las cuales se les brindara una hoja con las instrucciones precisas de como llevar un objeto de una pósicion 1 a una posicion 2.
En esta ocasion se hara uso de 2 tarjetas y una hoja de pápel las cuales son nuestras fichas fundamentales para llevar a cabo la actividad planteada

El objetivo sera que cada uno de los participantes logre con una sola de sus manos formar una piramide con las 2 tarjetas mencionadas anteriormente. Para esto a cada participante se le entrego una hoja con las siguientes instrucciones:
 \newline


1.) Desplazar la hoja de papel hacia la derecha hasta que las tarjetas sean visibles. \newline

2.) Levantar las dos tarjetas con una sola mano. \newline

3.) Asegurese de que las tarjetas esten unidas uniformemente entre si. \newline

4.) Sostenga la parte superior de las tarjetas con los dedos indice, pulgar y medio. \newline

5.) Haciendo uso del dedo meñique y el anular abra la parte inferior de las tarjetas sin soltar la parte superior. \newline

6.) Con las tarjetas ya en posicion recarguelas sobre la hoja de papel en posicion vertical. \newline

7.)Soltar las tarjetas levemente, asegurando la posicion final de una piramide.

\newpage

\section{Contenido} \label{contenido}


\subsection{Experiencia de cada voluntario}
Participante #1:\newline
Edad: 30 años  \newline

Con este participante se logra observar una ambiguedad en la instruccion #6 la cual dice : \newline
''Con las tarjetas ya en posicion recarguelas sobre la hoja de papel en posicion vertical'', esta instruccion no fue clara para el participante en un inicio, puesto que se dispuso a armar la piramide sobre otro lugar distinto al mencionado, se logra observar tambien que esta persona cumple a cavalidad el objetivo planteado tras leer detenidamente.\newline
\newline
Participante #2:\newline
Edad: 16 años  \newline

Con esta participante se observa de nuevo una ambiguedad en la instruccion #6 al igual que en la instruccion #3, siendo este el principal motivo de que la participante no lograra el objetivo planteado.\newline
\newline
Participante #3:\newline
Edad: 23 años  \newline

Este participante no tuvo problema alguno con las instrucciones, logrando su objetivo tras seguir a plenitud cada una de las instrucciones dadas, aunque esta persona aprovecho la ambiguedad de la instruccion #3 para recostar la tarjeta en la palma de su mano y lograr mas facil su objetivo.

\newpage 

\subsection{observaciones generales}\newline
Tras observar detenidamente cada uno de los participantes se observan ambiguedades en las instrucciones 3 y 6 las cuales afectaron el desarrollo de la actividad en algunos participantes.  \newline \newline
La actividad hubiera podido ser mas amena con el participante si se hubiera añadido una instruccion mas la cual permitiera el apoyo directo en la muñeca. \newline \newline
Todos los participantes hicieron uso de la superficie para apoyarse en la solucion, esto tampoco se aclaro directamente en las instrucciones previas. \newline \newline 
El tiempo promedio empleado para elk desarrollo de esta actividad es de aproximadamente 1 minuto, teniendo asi una buena eficiencia en el tiempo empleado. \newpage 



\subsection{conclusiones}
Se nota que hubo una eficiencia del 66.6 con las instrcucciones planteadas para el desarrollo de la actividad.2 de 3 participantes lograron el objetivo planteado.\newline 

Hubieron falencias en las instrucciones 3 y 6 puesto que los participantes encontraron ambiguedades las cuales jugaron en contra a la hora de lograr el objetivo.\newline 

A la hora de programar se sigue un proceso similar a lo planteado en esta actividad puesto que las instrucciones dadas tienen que ser claras para cualquier persona , no hay lugar para las ambiguedades pues esto nos ocasionara errores.\newline 

Las instrucciones siempre tienen que ser lo mas amigables con el usuario puesto que este sera quien pondra a prueba nuestro trabajo y vera errores que nosotros como programadores no vimos.



\end{document}
